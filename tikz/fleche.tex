\documentclass[11pt]{article}

\usepackage[utf8]{inputenc}
\usepackage[T1]{fontenc}
\usepackage{tikz}
\newcommand*\markterm[2]{%
	\tikz[remember picture,anchor=base west,baseline,inner sep=0pt, outer sep=0pt]\node(#1){$#2$};%
}
\newcommand*\arrowtoterm[3][]{%
	\tikz[remember picture,anchor=base west,baseline,inner sep=0pt, outer sep=0pt]\node(x){#3};%
	\tikz[remember picture,overlay,->,shorten <=2pt,shorten >=2pt,#1]\draw(x)to(#2);%
}
\begin{document}
\parindent=0pt
Voici une équation
\[3x^2-\markterm{a}{4x}+1\]
Dans cette équation, ce \arrowtoterm{a}{terme} est le terme de degré 1.

\bigbreak

Voici une équation
\[\markterm{c}{3x^2}-\markterm{b}{4x}+1\]
Dans cette équation, ce \arrowtoterm[out=90,in=90,thick,red,-stealth]{b}{terme} est le terme de degré 1 et \arrowtoterm[out=270,in=270,thick,blue,-stealth]{c}{celui-ci} est celui de degré 2.
\end{document}